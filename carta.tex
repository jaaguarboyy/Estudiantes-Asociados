\documentclass{article}
\usepackage[spanish]{babel}
\usepackage{amsmath}
\usepackage{amsfonts}
\usepackage{amssymb}


\begin{document}
\begin{flushright}
    29 de octubre de 2024
\end{flushright}
\noindent A quien corresponda. 
\vspace{5pt} \\ 
\noindent Me presento, soy Balam T. Fuentes V. de séptimo semestre 
de la carrera de filosofía, en la Facultad de Filosofía y Letras de la 
UNAM. Mis intereses de investigaciones que son realizadas en el instituto 
de investigación son: filosofía de la lógica, filosofía del lenguaje, 
filosofía de la mente y epistemología. En específico de la filosofía de la 
mente estoy interesado en el problema de la relación mente-mundo, 
para esto me he acercado un poco a las clases de Mario Goméz Torrente
y su propuesta de \textit{``razonamiento antrópico y multiversos''};
esto a su vez me ha servido para preguntarme sobre cómo el conocimiento 
se puede deducir de la relación mente-mundo.
Por otra parte, en filosofía de la lógica estoy interesado en lógica modal,
criterios de constancia lógica, teoría de modelos, teoría de conjuntos y pluralismo. 
Algunos de estos temas los conocí gracias al texto del Dr. Gómez Torrente  
\textit{Forma y Modalidad}. De igual manera, de pluralismo lógico he
indagado los trabajos del Dr. Barceló; \textit{Pluralismo lógico} (pero 
me gustaría hacer hincapié que el pluralismo en el que estoy pensando 
es restringido para no relativizar la lógica). De filosofía del lenguaje 
he leído, también, cosas del Dr. Torrente como \textit{Perceptual Variation, 
Color Language, and Reference Fixing. An Objectivist Account}, entre otros. 
Y en filosofía de la ciencia tengo cierto interés en ahondar en el tema 
de filosofía de la física. 
\vspace{5pt} \\ 
\noindent Ahora, los beneficios que me traería entrar al programa de 
estudiantes asociados del IIF son: un refuerzo más riguroso en mis 
conocimientos de las áreas mencionadas (es decir, perfeccionarlos), 
enfocarme en la especialidad de dichas áreas, además de aprender a 
investigar de una manera más rigurosa y específica, mejorar mi 
forma de escribir ensayos o artículos (es decir, organizar de mejor 
mis ensayos y de forma más análitica), también me ayudaría a mejorar 
mis presentaciones (en coloquios, seminarios, clases, etc.), 
me ayudaría a aprender a moderar (mesas, debates, seminarios, 
conversatorios, etc.), aprendería a organizar eventos de 
una forma adecuada. Finalmente, creo que mi estancia en el programa me 
ayudaría a formarme como investigador en el área de filosofía.
\vspace{5pt} \\ 
Dicho esto último, mi aporte dado al programa de estudiantes
asociados responderá a: organización de eventos y creación de seminarios 
(de lógica, lenguaje, ontología análitica, filosofía de las matemáticas, 
etc.. Algunos de los cuales me interesaría hacerlos interuniversitarios, si 
es que es posible.). Uso adecuado y correcto de las instalaciones. Además, 
me comprometo a la presentación y moderación en el seminario de estudiantes 
asociados y el seminario de investigadores.
\vspace{5pt} \\ 
\begin{flushright}
    Un saludo, \\
    Balam Fuentes
\end{flushright}


\end{document}

%CONCIDERA:
Una carta de exposición de motivos firmada por la persona 
candidata que muestre interés por la investigación realizada 
en el Instituto y en la que indique no sólo qué beneficios 
académicos le traerá el Programa, sino también qué puede 
aportar a este último.