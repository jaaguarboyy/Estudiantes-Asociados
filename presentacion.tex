\documentclass{beamer}
\usepackage[spanish]{babel}
%\usepackage[latin1]{inputenc}
\usepackage{multicol} % indice en 2 columnas
\usepackage{url}
\usepackage{graphicx}

\usetheme{Warsaw}
%\usecolortheme{crane}
\useoutertheme{shadow}
%\useinnertheme{}

\setbeamertemplate{navigation symbols}{} % quitar simbolitos

\title[Introducción]{Conocimiento de la matem\'atica desde enunciados de 
identidad}
\author[Balam T. Fuentes Villianueva]
{Balam T. Fuentes V.}
\institute[Universidad Nacional Autónoma de México]
{
  
  Facultad de Filosofía y Letras, UNAM.
  \and
  \texttt{bllvllnv@gmail.com}
}
\date{29 de noviembre de 2024}

\begin{document}
%------------------------------------------------------------
\frame{\titlepage}
%------------------------------------------------------------
\begin{frame}
  \frametitle{Índice}
  \tableofcontents
\end{frame}
%------------------------------------------------------------
\section{Introducci\'on}
    \subsection{Dilema de Benacerraf}
        \begin{frame}{Introducci\'on}
            \begin{itemize}
                \item[] En la filosofía de las matemáticas existe un dilema 
                epist\'emico. El reto epist\'emico es planteado por 
                Benacerraf en su texto \textit{Mathematical truth}. \pause

                \item[] Dicho reto, consiste en que para que haya un 
                conocimiento matemático tiene que cumplir con una 
                de dos cosas:
            \end{itemize}      
        \end{frame}
%------------------------------------------------------------
        \begin{frame}{Dilema de Benacerraf}
            \begin{block}{Dilema de Benacerraf}
                \noindent \small ``Es mi opinión que dos tipos muy distintos de 
                preocupaciones han motivado por separado consideraciones sobre 
                la naturaleza de la verdad matemática: 
                \begin{itemize}
                    \item (1) la preocupación por 
                    tener una teoría semántica homogénea en la que la semántica 
                    para las proposiciones de las matemáticas sea paralela a 
                    la semántica para el resto del lenguaje, y \pause
                    \item (2) la preocupación de que la exposición de la verdad 
                    matemática engrane una epistemología 
                    razonable'' (Benacerraf, 2004).
                \end{itemize} 
            \end{block}
        \end{frame}
%------------------------------------------------------------
        \begin{frame}{Dilema de Benacerraf}
            \begin{itemize}
                \item[] Aún así, Benacerraf conciderá que para que haya una 
                buena justificación de las verdades de la matemática 
                sería necesario que cumplan con ambas condiciones. \pause 

                \item[] El problema surge al considerar la semántica 
                tarskiana para las oraciones matemáticas, dicha semántica nos 
                compromete con la existencia de objetos matemáticos. \pause 

                \item[] El dilema surge al considerar nuestra epistemología, 
                dado que es necesario explicar cuál es el acceso 
                que tenemos a dichos objetos, el cual no puede ser causal
            \end{itemize}

        \end{frame}
%------------------------------------------------------------
    \subsection{Plan de trabajo}
        \begin{frame}{Plan de trabajo}
        \begin{itemize}
                \item [] A esto surge la pregunta:
            \begin{block}{Pregunta principal}
                ¿cómo podemos conocer las verdades 
                matemáticas? \pause 
            \end{block} 
        \end{itemize}
            La postura filosófica de Agustín Rayo nos ayudará 
                a responder a esta pregunta. Dihca 
                postura es la que presenta en \textit{La construcción 
                del espacio de posibilidades}. \pause   

            El plan de trabajo consistirá primero en 
            exponer tres argumentos de Rayo los cuales nos 
            ayudan a `disolver' el problema de Benacerraf.
        \end{frame}
%------------------------------------------------------------
    \section{La propuesta de Rayo}
      \begin{frame}{[Números]}
        \noindent Para obtener una buena epistemología de las 
        matemáticas Rayo da un ejemplo el cual denomina como:
        \begin{block}{[Números]}
            \pause ``[Números] \\ 
            Que el número de las $Fs$ sea $n$ es simplemente 
            que haya $n$ $Fs$'' (Rayo, 2013). \pause
        \end{block}
        Para observar como es que este ejemplo nos ayuda a `disolver' el 
        dilema de Benacerraf tenemos que revisar tres agumentos de Rayo. 
      \end{frame}
%------------------------------------------------------------
      \begin{frame}{Argumentos}
        Los argumentos que tenemos que revisar son tres, los cuales 
        son: \pause
        \begin{itemize}
            \item [] 
            \begin{block}{Argumentos}
                \begin{itemize}
                    \item [1] Composicionalismo + \textit{``es simplemente''} 
                    \pause 
                    \item [2] Epistemología de enunciados de identidad \pause
                    \item [3] Una semántica acorde a [Números] 
                \end{itemize}
            \end{block}
        \end{itemize}
      \end{frame}
%------------------------------------------------------------

      \subsection{Primer argumento}
      \begin{frame}{Primer argumento}
        Rayo en su texto menciona que de hecho no hay 
        razones lingüisticas para rechazar el ejemplo de [Números].
        \vspace{5pt} \pause \\
        Para esto tenemos que revisar la postura de composicionalismo 
        y la metafísicalista.
      \end{frame}
%------------------------------------------------------------
      \subsubsection{Metafísicalismo}
      \begin{frame}{Metafísicalismo}
        El metafísicalismo lo definiremos como: 
        \begin{block}{Metafísicalismo} \pause
            Es aquella postura que se compromete con que de 
            hecho hay una única y privilegiada manera de 
            representar el mundo. \pause
        \end{block}
        Es decir, el metafísicalismo no aceptaría el ejemplo de 
        [Números].
      \end{frame}
%------------------------------------------------------------
      \subsubsection{``Es simplemente''}
      \begin{frame}{Operador de Rayo}
        Hay dos posibles lecturas para el operador \textit{`es simplemente'}.
        \begin{block}{Es simplemente}
            \begin{itemize}
                \item Lectura simétrica: \pause
                ``I will be treating ‘just is’-staments as equivalent to the 
                corresponding ‘no difference’ statements, I will be treating 
                the ‘just is’-operator as symmetric'' (Rayo, 2013a) \pause
                \item Lectura asimética: \pause
                ``that a `just is’-statements should only be conted as true 
                if the right-hand-side `explains’ the right-hand-side, 
                or if it is in some sense `more fundamental’ '' (Rayo, 2013a)
            \end{itemize}
        \end{block}
      \end{frame}
%------------------------------------------------------------
      \subsubsection{Composicionalismo}
      \begin{frame}{Composicionalismo}
        El composicionalismo lo definiremos como: 
        \begin{block}{Composicionalismo} \pause
            Es aquella postura que se compromete con que se 
            puede representar al mundo de diferentes maneras y no 
            se compromete con que exista una única manera de representar 
            al mundo y que haya una representación privilegiada del mundo.
        \end{block}
      \end{frame}
%------------------------------------------------------------
      \subsection{Segundo argumento}
      \begin{frame}{Segundo argumento}
      En esté seguno argumento Rayo habla de la epistemología 
      de los enunciados de identidad. Sus principales objetivos son
      \begin{block}{Objetivos del segundo argumento} \pause
        \begin{itemize}
          \item  ¿en qué consiste la verdad de un enunciado de identidad? \pause
          \item ¿qué quiere decir que una concepción del espacio de posibilidad 
          sea objetivamente correcta? \pause
        \end{itemize}
      \end{block}
      Para esto necesitamos justificar cómo es que podemos 
      configurar el espacio de posibilidades y cómo es que 
      se satisfacen las condiciones de verdad de los enunciados 
      de identidad del tipo `ser simplemente'. 
      \end{frame}
%------------------------------------------------------------
      \begin{frame}{Definición del esapcio de posibilidad}
        Para comenzar necesitamos la definición del espacio de 
        posibildiad de Rayo la cual es: 
        \begin{block}{Espacio de posibilidad}\pause
          ``el espacio de posibilidades es el 
          conjunto de alternativas con las que trabajamos cuando nos 
          preguntamos cómo es el mundo'' (Rayo, 2013b).
        \end{block}
      \end{frame}
%------------------------------------------------------------
      \begin{frame}{Configuración del espacio de posibilidades}
        Para responder a las preguntas necesitamos saber como es 
        que se configura dicho espacio. 
        \begin{itemize}
          \item [Primero] Necesitamos aceptar los enunciados 
          de identidad del operador `es simplmente'.
          
          \item [] Ya que gracias a ellos empezamos la configuración de dicho 
          espacio. 
          \item[] Al aceptar los enunciados de identidad vamos recortando 
          el espacio de posibildiad, con el cuál estamos trabajando.
          \item[] Aceptar ó rechazar estos enunciados de identidad equivale 
          a una ventaja teórica; si es una desventaja los rechazaremos.  
        \end{itemize}
      \end{frame}
%------------------------------------------------------------
      \begin{frame}{Satisfacción de enunciados de identidad}
        La definición de satisfacción de enunciados de identidad 
        es la siguiente: 
        \begin{block}{Enunciados de idendetidad} \pause
          \small ``I shall say that a sentence’s 
          truth-conditions are trivial when the assumpion that they 
          fail to be satisfied would lead to absurdity. (In other 
          words: a sentence has trivial truth-conditions just is in 
          case its negation is metaphysically 
          inconsistent.)'' (Rayo, 2013a). \pause
        \end{block}
        Es decir, las condiciones de verdad serán triviales a menos 
        que de hecho pueda haber una contradicción 
        verdadera, lo cuál es algo absurdo.
      \end{frame}
%------------------------------------------------------------
      \begin{frame}{Satisfacción de enunciados de identidad}
        Por lo tanto podemos concluir que la epistemología de los 
        enunciados de identidad tiene que ver en c\'omo 
        es que se satisfacen dichos enunciados de identidad y que 
        aceptar estos implica que se garantiza una ganacia teórica 
        al respecto. 
        
        La \'unica manera de rechazarlos es cuando no 
        garantice una ganancia porque significaría tener una perdida 
        teórica. Además la verdad de estos enunciados es trivial 
        ya que tiene que coincidir con el mundo serán falsos en el
        caso en que el mundo sea contradictorio, lo cuál es falso.
      \end{frame}
%------------------------------------------------------------
      \subsection{Tercer argumento}
      \begin{frame}{Tercer argumento}
        Hace falta mostrar como es que Rayo propone una semántica que 
        vaya acorde con el ejmplo de [Números] y que haya una epistemología
        que respalde deicha semántica.
        \begin{block}{Tesis matemáticas}
          Para esto Rayo parte de dos la unión de dos tesis de la filosofía de 
          las matemáticas. \pause
          \begin{itemize}
            \item [1] \textit{platónismo matemático} \pause
            \item [2] \textit{trivialismo matemático}
          \end{itemize}
        \end{block}
      \end{frame}
%------------------------------------------------------------
      \begin{frame}{Definición de platónismo y trivialismo}
        La definición que Rayo ofrece es: 
        \begin{block}{Platónismo y trivialismo matemático}\pause
          \small``el \textit{platonismo matemático}, es 
          decir, la tesis de que existen objetos [\dots] \textit{trivialismo
          matemático}, [\dots], en otras palabras: no se requiere de nada del 
          mundo para que sea verdadera una verdad de las matemáticas 
          puras y se requeriría algo absurdo del mundo para que 
          fuera una falsedad del mundo para que fuera verdadera 
          una falsedad de las matemáticas puras'' (Rayo, 2013b).
        \end{block}
      \end{frame}
%------------------------------------------------------------
      \begin{frame}{Trivialismo platónico}
        \begin{itemize}
          \item Rayo defiende sus tesis de la siguiente manera. El trivialismo 
          lo defiende diciendo que no hay nada que evite que existan [Números]
          en el mundo porque la exigencia de eso es trivial.\pause

          \item El platonismo lo defiende de una manerá muy similar, solo que no 
          hace una reducción al absurdo, pero como la conclusión de la prueba 
          resulta siendo verdadera porque que no haya números implica que 
          exista el 0 siendo así lo mismo.

          \item Gracias a esto podemos agregar ciertos axiomas de manera trivial, 
          siempre y cuando al agregar estos enunciados matemáticos nuestro 
          espacio de posibilidad siga siendo consistente.  
        \end{itemize}
      \end{frame}
%------------------------------------------------------------
      \begin{frame}{Semántica}
        ¿Cuál es la semántica que Rayo propone para 
        que le haga justicia al ejemplo de [Números]?. Rayo 
        para esto plantea una semántica donde dicha semántica 
        empieza con que el lenguaje se basa en la siguiente manera de 
        denotar las fórmulas: 
        \begin{block}{Semántica} \pause
          \begin{itemize}
            \item ``Consideremos la notación `[\dots]$_w$'
            que se lee `es verdad que en el mundo posible \dots' '' (Rayo, 2013b).

            \item Después agrega una semántica trivialista 
            la cual Rayo dice que se expresa así: \#$x$ (Números $x$)= 0.

            \item Dónde está semántica cumple con una asignación interna y una externa.
            La cual se va a denotar de la siguiente manera: ``el número de 
            las $z$ tales que [$z$ es un número]$_w$ = 0'' y ``[el número de 
            las $z$ tales que $z$ es un número = 0]$_w$''.
          \end{itemize}
        \end{block} 
      \end{frame}
%------------------------------------------------------------
      \begin{frame}{Semántica}
        \begin{itemize}
          \item [] Donde la primera manera es la asignación externa y la segunda 
          es la asignación externa. Ambas son muy parecidas y son 
          denotaciones metalingü\'isticos que denotan lo mismo de diferentes 
          formas, pero gracias a nuestro operador es que podemos 
          ponerlos en una identidad.

          \item[] Rayo aclara que no se 
          presupone que en estos ejemplos de hecho haya 0 números. 
          A lo que quiere apuntar es que como filósofos/teóricos 
          podemos tener las herramientas y la capacidad de hablar de números 
          o cualquier objeto que satisface con dichas propiedades en cualquier
          $w$.
        \end{itemize}
      \end{frame}
%------------------------------------------------------------
      \begin{frame}{Semántica}
        Rayo dice que el trivialismo plat\'onico de hecho no 
        reconoce al dilema de Benacerraf como un dilema genuino, ya 
        que parte de que no necesitamos nada del mundo para que de hecho 
        existan los números. 
        
        De hecho si nos damos cuenta lo que pasa 
        es que todo lo que ha propuesto Rayo satisface ambas condiciones 
        que propone Benacerraf, siendo así que disuelve el problema. 
        Pero lo que plantea Rayo es que no porque el trivialismo 
        platónico disuelva esto signifique no se tiene que justificar.
      \end{frame}
%------------------------------------------------------------
    \begin{frame}{Logros cognitivos}
      Él justifica este hecho diciendo que el conocimiento matemático 
      que podemos conseguir desde esta tesis se convierten en logros 
      cognitivos.

      Se convierten en logros cognitvos porque como 
      las condiciones de verdad de los enunciados de identidad son 
      triviales solo necesitamos dos cosas para justificar esto.
      \begin{block}{Logros cognitivos}
        \begin{itemize}
          \item [(i)] Saber c\'omo es que funciona la adquisición de información 
          del mundo y
          \item [(ii)] saber c\'omo es que funciona esta adquisición 
          y como somos capaces de utilizarla.
        \end{itemize}
      \end{block}
    \end{frame}
%------------------------------------------------------------
    \begin{frame}{Logros cognitivos}
      En la primer parte, podemos decir que la adquisición de conocimiento
      cuando suponemos por ejemplo que aceptamos ciertos 
      axiomas de la matemática.

      En la parte de las deducciones lo que hacemos es saber 
      como es que ciertas verdades se siguen por consecuencia 
      lógica.
    \end{frame}
%------------------------------------------------------------
  \section{Conclusión}
    \begin{frame}
      Como expusimos a lo largo de esta ponencia podemos responder la 
      pregunta principal diciendo que los 
      enunciados de identidad nos ayudan a configurar el espacio 
      de posibilidad donde las condiciones de satisfacción
      de verdad son triviales porque no hay nada en el mundo 
      que evite que exsistan números, siendo así que de hecho 
      estos conocimientos son logros cognitivos. 
    \end{frame}
%------------------------------------------------------------
  \section{Conclusión}
    \begin{frame}
      La respuesta es bastante satisfactoria, pero esta misma 
      respuesta arroja otras preguntas, algunas de ellas 
      son: 
      \begin{block}{Problematicas}
        \begin{itemize}
          \item ¿los logros cognitivos necesariamente son 
          genuinos?
          \item ¿esto manera de ver el conocimiento matemática no 
          cae en el problema de la omniciencia lógica?
        \end{itemize}
      \end{block}
      Pero dichas preguntas se abordarán en otro trabajo. 
    \end{frame}
%------------------------------------------------------------
\section{Bibliograf\'ia}
    \begin{frame}{Bibliografía}
      \begin{itemize}
        \item Benacerraf, P. (1973). Mathematical truth. \textit{
            the Journal of philosophy
        }, 70(19), 661-679.
        \item Benacerraf, P. (2004). La verdad matemática. \textit{
            Biblioteca Digital del ILCE
        }. 
        \item Elga A, y Rayo, A. (2004). Fragmetation and logical 
        omniscience. \textit{No\^us}, 56(3), 716-714.
        \item Frege, G. (2016). Escritos sobre lógica, semántica y 
        filosofía de las matemáticas, trad. \textit{X. de Donato, 
        C.U. Molines, H. Padilla y M. Valdés, UNAM-IIF, Ciudad de México.}  
        \item Orayen, R. (1991). La lógica y el dilema de Benacerraf. \textit{
            Crítica: Revista Hispanoamericana de Filosofía
        }, 127-138.
      \end{itemize}
    \end{frame}
      \begin{frame}{Bibliografía}
        \begin{itemize}
            \item Pritchard, D. (2009). Knowledge, understanding and epistemic 
        value. \textit{Royal Institute of Philosophy Supplements}, 64 , 19–43.
        \item Pritchard, D. (2016). Epistemic risk. \textit{The Journal of 
        Philosophy}, 113 (11), 550–571.
        \item Rayo, A. (2008). On specifying truth-conditions. \textit{Philosophical 
        Review}, 117 (3), 385–443.
        \item Rayo, A. (2013a). \textit{The construction of logical space}. OUP Oxford
        \item Rayo, A. (2013b). \textit{La construcción del espacio de posibilidad}. 
        México: IIF/UNAM.
        \item Stalnaker, R. (1991). The problem of logical omniscience, i. 
        \textit{Synthese}, 425–440.
        \item Stalnaker, R. (1978). Assertion. \textit{En Pragmatics} (pp. 315–332). 
        Brill.
        \end{itemize}
    \end{frame}
\end{document} 