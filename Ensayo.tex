\documentclass[]{article}
\usepackage[spanish]{babel}
\usepackage{amsmath}
\usepackage{amsfonts}
\usepackage{amssymb}
\usepackage{multicol}
\usepackage{apacite}
\bibliographystyle{apacite}

\title{Conocimientos de la matemática 
desde enunciados de identidad}
\author{\large Balam Fuentes}

\begin{document}
\maketitle

\noindent 
En la filosofía de las matemáticas existe un dilema epist\'emico. 
El reto epist\'emico es planteado por Benacerraf en su texto
\textit{Mathematical truth}\footnote{\cite{benacerraf1973mathematical}}.
El cuál a sido valorado como un problema v\'alido y genuino por varios 
autores uno de ellos es Orayen en \cite{orayen1991logica}.
Dicho reto, consiste en que para que haya un conocimiento 
matemático tiene que cumplir con una de dos cosas.
\begin{quotation}
    \noindent \small ``Es mi opinión que dos tipos muy distintos de 
    preocupaciones han motivado por separado consideraciones sobre 
    la naturaleza de la verdad matemática: (1) la preocupación por 
    tener una teoría semántica homogénea en la que la semántica 
    para las proposiciones de las matemáticas sea paralela a 
    la semántica para el resto del lenguaje, y (2) la preocupación 
    de que la exposición de la verdad matemática engrane una 
    epistemología razonable'' \cite{benacerraf2004verdad}.
\end{quotation}  

Aún así, Benacerraf conciderá que para que haya una 
buena justificación de las verdades de la matemática 
sería necesario que cumplan con ambas condiciones. El problema 
central surge al considerar la semántica tarskiana para las 
oraciones matemáticas, dicha semántica nos compromete con la 
existencia de objetos matemáticos, lo cual por sí mismo no 
parece ser un problema. El dilema surge al considerar nuestra 
epistemología, dado que es necesario explicar cuál es el acceso 
que tenemos a dichos objetos, el cual no puede ser causal, dado 
que los objetos matemáticos son objetos abstractos.
El autor considera esto porque cualquier tipo de 
explicación que no considere a las dos, fracasa. Si  
bien pueden haber explicaciones del discurso de la 
matemática y otras no 
ambas fracasan ya que, según Benacerraf, carecen de 
una buena explicación en c\'omo es que se obtiene 
conocimiento matemático \cite{benacerraf2004verdad}. 
En cambio, las justificaciones que parten de enunciados matemáticos
que asumen verdades matemáticas; las cuales no conectan 
con el análisis de la verdad en cualquier condición de estos 
enunciados. En este último ejemplo se refiere 
principalmente a axiomas matemáticos. 

A esto surge la pregunta ¿c\'omo podemos conocer las verdades 
matemáticas? La postura filos\'ofica de Agustín Rayo nos ayudará 
a responder a esta pregunta. Para poder dar una 
respuesta a dicha pregunta nos basaremos en
los textos Rayo, los cuales serán:  
\textit{The Construction of the logical space}, \textit{On 
specifying truth-conditions}, entre otros. La elección de estos textos 
es porque él explica c\'omo es que su propuesta del 
espacio de posibilidad y su operador \textit{``es 
simplemente''} puede ayudarnos a conocer las verdades 
matemáticas, disolviendo así el dilema de Benacerraf. 
Con esto en mente y como ya expusimos el problema de Benacerraf
lo segundo que tenemos que exponer es la propuesta 
de Agustín Rayo. Por último, podemos explicar qu\'e problemas 
surgen con dicha solución.   

\subsection*{La propuesta de Agustín Rayo}
\noindent Para obtener una buena epistemología de las 
matemáticas Rayo da un ejemplo el cual denomina como 
[Números],\footnote{ 
    ``[Números] 
    Que el número de las $Fs$ sea $n$ es simplemente 
    que haya $n$ $Fs$.'' \cite{rayo2013} 
} este ejemplo es importante ya que nos permite 
resolver o disolver el problema de Benacerraf; ya antes planteado. 
Nos ayuda a `resolver', o dicho de otra manera `disolver', 
dicho dilema. Para poder plantear esto 
\textit{el primer argumento} de Rayo es que no hay razones lingüisticas para 
rechazar el ejemplo anteriormente mencionado. \textit{La segunda parte} del 
argumento de Rayo es sobre la epistemología de los enunciados de 
identidad. \textit{Por último} Rayo necesita una semántica que vaya acorde 
con el ejemplo de [Números] y una epistemología que también 
vaya acorde con el ejemplo. 


\subsection*{\small Primer argumento}

Por lo tanto, necesitamos primero explicar la primera parte del 
argumento de Rayo. Él argumenta que para justificar 
que no hay razones lingüísticas para rechazar el ejemplo puede haber 
diferentes maneras de represtnar, de manera gramatical, un mismo hecho 
del mundo y expresarán lo mismo. Rayo parte del parágrafo 8 de la 
conceptografía de Frege\footnote{\cite{frege2016escritos}}. Es decir 
puede haber dos oraciones que tienen formas diferentes pero 
tengan el mismo contenido. 

Para lograr este objetivo Rayo crítica la tesis del \textit{metafísicalista} 
y propone que la tesis que nos permita justificar el prop\'osito de
su primer argumento es el \textit{composicionalismo} y su operador
\textit{`es simplemente'}. El metafísicalismo, es aquella postura 
que se compromete con que de hecho hay una única y privilegiada 
manera de representar el mundo. Es decir, que ellos no aceptarían 
el ejemplo de Rayo de [Números].\footnote{Vease la nota de pie de 
pagína 2 o vease \cite{rayo2013} cápitulo 1.} Ellos no aceptarían 
dicho enunciado ya que primero no aceptarían que puede 
haber enunciados de identidad que pongan en simetría dos hechos 
del mundo que representen lo mismo, es decir; ellos no aceptarían 
el operador \textit{`es simplemente'}. Muy similar al primero,
la segunda razón por la cual no aceptarían el ejemplo [Números] 
es porque considerar que la oración ``que el número de las $Fs$
sea $n$'' no es lo mismo que ``que haya $n$ $Fs$'' \cite{rayo2013}. 
En otras palabras, no aceptar\'ian que dos enunciados que hablen de 
lo mismo, o tengan el mismo contenido, porque ellos no aceptarían que 
dos enunciados con diferentes formas gramaticales sean lo mismo. 

Antes de mencionar el composicionalismo falta explicar 
que es el operador \textit{`es simplemente'}. Hay dos posibles 
lecturas para el operador \textit{`es simplemente'}. La 
inadecuada es la asmim\'etrica y la sim\'etrica. La que le interesa a 
Rayo en su propuesta es la sim\'etrica ya que en palabras de él
``I will be treating ‘just is’-staments as equivalent to the 
corresponding ‘no difference’ statements, I will be treating 
the ‘just is’-operator as symmetric''\cite{rayo2013construction}. 
Es decir, que no hay diferencia entre los enunciados que se ponen 
a considerar. O sea, son enunciados de identidad. Con identidad
Rayo se refiere a una manera en espec\'ifica ya que no considera 
que `es simplemente' sea lo mismo que `='; sino que las 
oraciones pueden tener formas gramaticales diferentes pero 
pueden expresar lo mismo. Un ejemplo puede salir de la misma 
conceptografía ``los persas fueron derrotados en Platea por los 
griegos'' \textbf{`es simplemente'} ``los griegos derrotaron a los 
persas en Platea'' \cite{frege2016escritos}. Dicho ejemplo 
puede funcionar porque representa un mismo hecho del mundo 
de dos maneras diferentes pero ambas oraciones se ponen en simetría.
En cambio, la manera inadecuada de leer el operador de Rayo es 
la siguiente ``that a `just is’-statements should only be conted as true 
if the right-hand-side `explains’ the right-hand-side, or if it is in 
some sense `more fundamental’ '' \cite{rayo2013construction}. 
Es decir, que si uno considera que las oraciones del tipo `ser simplemente'
el lado derecho explica el lado izquierdo es la lectura incorrecta 
ya que vuelven asmim\'etrico al operador, un ejemplo puede ser; 
que haya matemática `es simplemente' que exista un conjunto de $n$ 
números y que cumpla con tal y cual propiedades. Esa es la lectura 
incorrecta porque vuelve al operador asmim\'etrico. 

Ahora bien, ¿qué es el composicionalismo y como el operador de Rayo
es de suma importancia en esta postura?. El composicionalismo 
se puede definir, a diferencia del metafísicalismo, en que de hecho se
puede representar al mundo tal cual es de diferentes maneras y no 
se compromete con que exista una única manera de representar al mundo 
y que haya una representación privilegiada del mundo. Es decir, 
el composicionalismo puede aceptar enunciados de identidad del tipo 
`ser simplemente' ya que como se compromete a aceptar diferentes 
maneras de representar el mundo de un mismo hecho, independientemente 
de su forma gramatical, es capaz de aceptar que exista un operador que 
puede poner en simetría estas formas de representar los hechos del 
mundo. Además la postura composicionalista como ya nos dimos 
cuenta est\'a comprometida con que puede haber diferentes maneras 
de represenetar el mismo hecho del mundo, esto implica que a su 
vez el composicionalismo est\'a comprometido con una metafísca del 
mundo, es decir en tanto que el mundo es de la forma $x$ o $y$ 
tiene que garantizar que de hecho estas diferentes maneras de 
representar al mundo sean metafíscamente compatibles con el 
mundo, en otras palabras; que de hecho no sean absurdas o 
contradictorias esas maneras de representar.

Ya con esto en mente, podemos concluir que la relevancia del 
operador de Rayo es relevante para el composicionalismo 
porque es capaz de poner en sim\'etria dos formas de represenetar 
un hecho del mundo y que esto sea verdadero. Por lo tanto 
no hay razones lingüisticas para que rechacemos el ejemplo [Números].

\subsection*{\small Segundo argumento}
El segundo argumento de Rayo es sobre la epistemología de 
los enunciados de identidad. En este argumento de Rayo 
se pregunta por dos cosas ``¿en qué consiste la verdad de un
enunciado de identidad? y ¿qué quiere decir que una concepción
del espacio de posibilidad sea objetivamente correcta?'' \cite{rayo2013}.
Para esto Rayo necesita justificar c\'omo es que podemos 
configurar el espacio de posibilidades y c\'omo es que 
podemos satisfacer las condiciones de verdad de los enunciados 
de identidad del tipo `ser simplemente'. 

Pero a todo esto, ¿qué es el espacio de posibilidad según Rayo?
A esto Rayo dice ``el espacio de posibilidades es el 
conjunto de alternativas con las que trabajamos cuando nos 
preguntamos cómo es el mundo'' \cite{rayo2013}. Entonces, 
la respuesta que necesitamos para ambas respuestas es 
saber c\'omo es que se configura el espacio de posibilidad, 
al hacer esto sabremos c\'omo es que se satisfacen los 
enunciados de identidad dentro de los espacios de posibilidad. 
Ahora bien, para configurar el espacio de posibilidad 
necesitamos ser capaces de aceptar los enunciados de lo que estamos
hablando, es decir; que si aceptamos el ejemplo de [Números], 
entonces aceptamos ambos enunciados, es decir aceptamos 
que  ``que el número de las $Fs$ sean $n$'' y aceptamos también 
``hay $n$ $Fs$''\footnote{\cite{rayo2013}}. Al aceptar 
los enunicados y si aceptamos el enunciado de idendetidad que propone 
Rayo pasa que vamos configurando el espacio de posibilidad 
ya que vamos eliminando cierto espacio teórico que no nos 
interesa.\footnote{
    S\'i nos damos cuenta parece que Rayo en esta idea 
    est\'a siguiendo a su maestro Stalnaker, en espec\'ifico en
    la idea que él presenta en \cite{stalnaker1978assertion}.
} Un ejemplo de como es que vamos configurando el espacio de 
posibilidad es que podamos aceptar el siguient enunciado de 
de identidad: [Rojo] Que María exprerimente la sensación
de ver rojo es simplemente que María esté en el esto 
cerebal Rojo.\footnote{\cite{rayo2013}} Si nos damos cuenta 
al aceptar este enunciado de identidad estamos recortando el 
espacio teórico que nos interesa ya que no nos interesa si 
María esta experimentando el color azul, o cualquier otro 
tipo de color, sino que nos estamos conciderando unicamente 
la sensación de experimetar un color. Lo que dice 
Rayo a esto es que podemos aceptar o rechazar ciertos enunciados 
de identidad mientras esto equivalga a que agregue una ventaja
teórica de nuestro conocimiento, en el caso de aceptar 
un enunciado de identidad; en el caso contrario de 
rechazar equivaldrá a que aceptar dicho enunciado 
sería una desventaja para nuestro conocimiento 
teórico, por eso lo rechazamos. 

Para satisfacer las condiciones 
de verdad de los enunciados de identidad Rayo dice que las 
condiciones verdad son requisitos del mundo \cite{rayo2013}
y de hecho también dice: 
\begin{quotation}
    \noindent \small ``I shall say that a sentence’s 
    truth-conditions are trivial when the assumpion that they 
    fail to be satisfied would lead to absurdity. (In other 
    words: a sentence has trivial truth-conditions just is in 
    case its negation is metaphysically 
    inconsistent.)'' \cite{rayo2013construction}. 
\end{quotation}
Es decir, que aceptemos que el mundo de hecho puede ser de 
muchas maneras y se pueden representar de diferentes
maneras, en otras palabras; Rayo sigue mucho la propuesta del 
composicionalismo. Además de que las condiciones de verdad 
serán triviales a menos que de hecho pueda haber una contradicción 
verdadera, lo cuál es algo absurdo. Un ejemplo de esto 
puede ser: que tu est\'es aquí `es simplemente' que tú existas;
esto es algo trivial ¿no? ya que si no fuera verdadero de hecho 
tendríamos que llegar aquí para decir que no es trivial. Es
decir, para que esto sea falso tendríamos que decir que es
verdadero que estes aquí y que no est\'es aquí. Lo cual de hecho 
es absurdo porque esto no concuerda con c\'omo de hecho es el 
mundo; ya que el mundo en este caso tendría que ser inconsistente 
cosa que no es.\footnote{Rayo también dice 
más sobre este tema en su texto \cite{rayo2008specifying}.}

Por lo tanto podemos concluir que la epistemología de los 
enunciados de identidad tiene que ver en c\'omo 
es que se satisfacen dichos enunciados de identidad y que 
aceptar estos implica que se garantiza una ganacia teórica 
al respecto. La \'unica manera de rechazarlos es cuando no 
garantice una ganancia porque significaría tener una perdida 
teórica. Además la verdad de estos enunciados es trivial 
ya que tiene que coincidir con el mundo serán falsos en el
caso en que el mundo sea contradictorio, lo cuál es falso.

\subsection*{\small Tercer argumento}
Nos hace falta mostrar c\'omo es que Rayo propone 
una semántica que vaya acorde con el ejemplo de [Números] 
y una epistemología que de hecho respalde dicha semántica para 
poder decir que se disuelve el problema de Benacerraf. 

Para esto Rayo parte de dos la unión de dos tesis de la filosofía de 
las matemáticas. La primera tesis es el \textit{\textbf{platonismo 
matemático}}, la segunda tesis es el \textit{\textbf{trivialismo 
matemático}}.\footnote{Es decir, Rayo quiere defender la tesis 
del trivialismo plat\'onico.
} El platonismo matemático y el trivialismo matemático 
en palabras de Rayo es: 
\begin{quotation} 
    \noindent \small``el \textit{platonismo matemático}, es 
    decir, la tesis de que existen objetos [\dots] \textit{trivialismo
    matemático}, [\dots], en otras palabras: no se requiere de nada del 
    mundo para que sea verdadera una verdad de las matemáticas 
    puras y se requeriría algo absurdo del mundo para que 
    fuera una falsedad del mundo para que fuera verdadera 
    una falsedad de las matemáticas puras'' \cite{rayo2013}.
\end{quotation} 

Rayo defiende sus tesis de la siguiente manera. El trivialismo 
lo defiende diciendo que no hay nada que evite que existan [Números]
en el mundo porque la exigencia de eso es trivial. Es decir, 
sí suponemos por reducción al absurdo que no existen números 
pasa que al no haber ningún número implica que hay 0 números y 
el cero es un número, por lo tanto hay una contradicción y de 
esto se sigue que los números de hecho existen; diciendo así 
que de hecho el trivialismo matemático es verdadero.
El platonismo lo defiende de una manerá muy similar, solo que no 
hace una reducción al absurdo, pero como la conclusión de la prueba 
resulta siendo verdadera porque que no haya números implica que 
exista el 0 siendo así lo mismo. La diferencia es muy sutil,
pero la diferencia parte de la reducción al absurdo o no. 
Al hacer esto a Rayo le permite decir que gracias al trivialismo 
podemos agregar ciertos axiomas de manera trivial, siempre 
y cuando al agregar estos enunciados matemáticos nuestro 
espacio de posibilidad siga siendo consistente.   

Ahora bien, ¿cuál es la semántica que Rayo propone para 
que le haga justicia al ejemplo de [Números]?. Rayo 
para esto plantea una semántica donde dicha semántica 
empieza con que el lenguaje se basa en la siguiente manera de 
denotar las fórmulas: ``Consideremos la notación `[\dots]$_w$'
que se lee `es verdad que en el mundo posible \dots' ''
\cite{rayo2013}. Después agrega una semántica trivialista 
la cual Rayo dice que se expresa así: \#$x$ (Números $x$)= 0.\footnote{
Revise el ejemplo de [Dinosaurios] dado en el texto de Rayo.
} D\'onde est\'a semántica cumple con una asignación interna y una externa.
La cual se va a denotar de la siguiente manera: ``el número de 
las $z$ tales que [$z$ es un número]$_w$ = 0'' y ``[el número de 
las $z$ tales que $z$ es un número = 0]$_w$''.
Donde la primera manera es la asignación externa y la segunda 
es la asignación externa. Ambas son muy parecidas y son 
denotaciones metalingü\'isticos que denotan lo mismo de diferentes 
formas, pero gracias a nuestro operador es que podemos 
ponerlos en una identidad. Ahora bien, Rayo aclara que no se 
presupone que en estos ejemplos de hecho haya 0 números. 
A lo que quiere apuntar es que como filósofos/teóricos 
podemos tener las herramientas y la capacidad de hablar de números 
o cualquier objeto que satisface con dichas propiedades en cualquier
$w$. Ahora bien, por cuestiones de espacio no explicaremos
a fondo dicho lenguaje, sino que \'unicamente abordarémos
los principios teóricos de dicho lenguaje; en 
dado caso revise el apéndice que aparece en \cite{rayo2013}.

Ahora, Rayo dice que el trivialismo plat\'onico de hecho no 
reconoce al dilema de Benacerraf como un dilema genuino, ya 
que parte de que no necesitamos nada del mundo para que de hecho 
existan los números. De hecho si nos damos cuenta lo que pasa 
es que todo lo que ha propuesto Rayo satisface ambas condiciones 
que propone Benacerraf, siendo así que disuelve el problema. 
Pero lo que plantea Rayo es que no porque el trivialismo 
plat\'onico disuelva esto signifique no se tiene que justificar. 

Él justifica este hecho diciendo que el conocimiento matemático 
que podemos conseguir desde esta tesis se convierten en logros 
cognitivos. Se convierten en logros cognitvos porque como 
las condiciones de verdad de los enunciados de identidad son 
triviales solo necesitamos dos cosas para justificar esto. 
(1) Saber c\'omo es que funciona la adquisición de información 
del mundo y (2) saber c\'omo es que funciona esta adquisición 
y como somos capaces de utilizarla. En la primer parte, 
podemos decir que la adquisición de conocimiento
cuando suponemos por ejemplo que aceptamos ciertos 
axiomas de la matemática. Cuando alguien acepta 
la verdad de ciertos axiomas adquiere cierto conocimiento
lingüístico donde dicha información será relevante 
para nuestros usos que le demos en contextos donde 
podamos responder a cuales son las propiedades 
de la verdad de un axioma. En la parte de las 
deducciones lo que hacemos es saber 
como es que ciertas verdades se siguen por consecuencia 
lógica. Es decir, que asumimos que la verdad de las 
cosas se pueden transmitir, o sea que existe una 
transmición de información.     

\subsection*{Conclusión}
Como expusimos a lo largo del texto podemos responder la 
pregunta principal del ensayo diciendo que los 
enunciados de identidad nos ayudan a configurar el espacio 
de posibilidad donde las condiciones de satisfacción
de verdad son triviales porque no hay nada en el mundo 
que evite que exsistan números, siendo así que de hecho 
estos conocimientos son logros cognitivos. 

La respuesta es bastante satisfactoria, pero esta misma 
respuesta arroja otras preguntas, algunas de ellas 
son ¿los logros cognitivos necesariamente son 
genuinos? Esta pregunta la podemos problematizar 
con los textos de Pritchard 
\cite{pritchard2016epistemic}, \cite{pritchard2009knowledge}
, entre otros. Otra pregunta que surge de esto es 
¿esto manera de ver el conocimiento matemática no 
cae en el problema de la omniciencia lógica?. La pregunta 
anterior la han intentado responder Rayo junto con Elga 
\cite{elga2022fragmentation} también lo ha intentado 
Stalnaker en su texto \cite{stalnaker1991problem}. 
Pero esto se dejará para un trabajo posterior.


\bibliography{ref}

\end{document}
%Concidera: 
Un trabajo muestra de a lo sumo 4,000 palabras sobre 
algún tema filosófico de su interés. De preferencia, 
que el texto sea una versión revisada de un muy buen 
trabajo final para alguna clase o algún texto ya 
publicado.